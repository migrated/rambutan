% Macros for processing javaweave output.

% This version is for embedding (in Plain Tex or LateX) and does not
% implement index or contents.

%---------------PRELIMINARIES-------------------------------------------
									
\def\title{} % can be set by the user

\let\maybe=\iftrue % Default is print all sections

\parskip 0pt

\parindent 1em  %  Also used for first line of code

\newif\ifpdf
\expandafter\ifx\csname pdfoutput\endcsname\relax \else\pdftrue\fi

\def\cmylink{0 1 0} % Link color in pdftex

%---------------FONTS---------------------------------------------------

\font\eightrm=cmr8
\font\sc=cmcsc10
\let\mainfont=\tenrm
\font\fourteenrm=cmr10 scaled \magstep2   \let\titlefont\fourteenrm
\font\fourteensc=cmcsc10 scaled \magstep2
\font\tentex=cmtex10 % TeX extended character set (used in strings)
\let\idfont\it
\let\reservedfont\bf

\def\\#1{\leavevmode\hbox{\idfont#1\/\kern.05em}} % italic type for identifiers
\def\|#1{\leavevmode\hbox{$#1$}} % one-letter identifiers look better this way
\def\&#1{\leavevmode\hbox{\reservedfont#1\/}} % boldface type for reserved words
\def\.#1{\leavevmode\hbox{\tentex % typewriter type for strings
  \let\\=\BS % backslash in a string
  \let\'=\RQ % right quote in a string
  \let\`=\LQ % left quote in a string
  \let\~=\TL % tilde in a string
  \let\ =\SP % space in a string
  \let\_=\UL % underline in a string
  \let\&=\AM % ampersand in a string
  #1}}

%---------------ABBREVIATIONS-------------------------------------------

\let\amp=\&
\let\SS=\S
\let\PP=\P

\let\Y\smallskip

\def\#{\hbox{\tt\char`\#}} % parameter sign
\def\${\hbox{\tt\char`\$}} % dollar sign
\def\%{\hbox{\tt\char`\%}} % percent sign
\def\^{\ifmmode\mathchar"222 \else\char`^ \fi} % pointer or hat
    % circumflex accents can be obtained from \^^D instead of \^
\def\AT!{@} % at sign for control text
\def\@{@} % at sign in strings

\chardef\AM=`\& % ampersand character in a string
\chardef\BS=`\\ % backslash in a string
\chardef\LB=`\{ % left brace in a string
\def\LQ{{\tt\char'22}} % left quote in a string
\chardef\RB=`\} % right brace in a string
\def\RQ{{\tt\char'23}} % right quote in a string
\def\SP{{\tt\char`\ }} % (visible) space in a string
\chardef\TL=`\~ % tilde in a string
\chardef\UL=`\_ % underline character in a string

\def\O#1{% octal, hex or decimal constant
  {\def\?{\kern.2em}%
  \def\${\ell}% long constant
  \def\_{\cdot 10^{\aftergroup}}% power of ten
  \def\~{\hbox{\rm\char'23\kern-.2em\it\aftergroup\?\aftergroup}}% octal
  \def\^{\hbox{\rm\char"7D\tt\aftergroup}}% double quotes for hex constant
  #1}}


\def\defin#1{\global\advance\ind by 2 \1\&{#1 }}
\def\D{\defin{define}}
\def\F{\defin{format}}

\def\J{\.{@\&}} % TANGLE's join operation

\def\=#1{\kern2pt\hbox{\vrule\vtop{\vbox{\hrule
        \hbox{\strut\kern2pt\.{#1}\kern2pt}}
      \hrule}\vrule}\kern2pt} % verbatim string
\def\lapstar{$^{\dag}$}
\let\*=\lapstar

\def\DO{\hbox{\sl\char'044}} % slant dollar sign
\let\G=\ge % greater than or equal sign
\def\H{{\rm\char'136}} % hat
\let\I=\ne % unequal sign
\let\K=\gets % left arrow
\let\L=\le % less than or equal sign
\let\R=\lnot % logical not
\let\S=\equiv % equivalence sign
\let\TI\sim % tilde
\let\V=\lor % logical or
\let\W=\land % logical and
\def\vert{|}

%---------------PDF LINKS-----------------------------------------------

\def\LP#1{#1}
\def\dest#1{}
\def\outline#1#2{}
\def\url{\bgroup\catcode`~=12 \furl}
\def\furl#1{{\tt#1}\egroup}

\ifpdf
   \pdfoutput=1
   \def\cmycolor#1{\pdfliteral{#1 0 k}}
   \def\LP#1{\pdfstartlink attr{/Border [0 0 0]} goto num #1
             \cmycolor\cmylink#1\pdfliteral{0 0 0 1 k}%
             \pdfendlink}
   \def\dest#1{\pdfdest num #1 xyz }
   \def\outline#1#2{\pdfoutline goto num #1 {#2}}
   \def\furl#1{\pdfstartlink attr{/Border [0 0 0]}%
               user{/Subtype /Link /A 
                    << /Type /Action /S /URI /URI (http://#1) >>}%
               \cmycolor\cmylink{\tt#1}\pdfliteral{0 0 0 1 k}%
               \pdfendlink\egroup}
   \ifx\pdfstartlink\undefined\let\pdfstartlink\pdfannotlink\fi
\fi

%---------------SECTIONS------------------------------------------------

\newif\ifon  \def\onmaybe{\let\ifon=\maybe} 

\outer\def\M#1.{\MN#1.\vfil\penalty-100\vfilneg\bigskip
                \startsection\enspace\ignorespaces}

\outer\def\N#1.#2.{\MN#1.\vfil\eject
  \def\rhead{\uppercase{\ignorespaces#2}}
  \message{*\modno}\outline{\modno}{\modno. #2}
  \writeconitem{#2}\modno
  \startsection{\bf\ignorespaces#2.}\quad\ignorespaces}

\def\MN\LP#1#2.{\par % common code for \M, \N
  {\xdef\modstar{#1#2}\let\*=\empty\xdef\modno{#1#2}}
  \edef\setdest{\dest{#1}}
  \ifx\modno\modstar \onmaybe \else\ontrue \fi
  \ifon\else\setdest\fi
  \ifon\mark{{\tensy x}\modno}}


\def\startsection{\Q\setdest\noindent{\let\*=\lapstar\bf\modstar.\enspace}}
\def\rhead{}

\def\X#1:#2\X{\ifmmode\gdef\XX{\null$\null}\else\gdef\XX{}\fi
  \XX$\langle\,$#2{\eightrm\kern.5em #1}$\,\rangle$\XX}

\def\XF#1:#2\XF{\ifmmode\gdef\XX{\null$\null}\else\gdef\XX{}\fi
  \XX{\tt(#2{\eightrm\kern.5em #1})}\XX}

\def\note#1#2.{\Y\noindent{\hangindent2em\baselineskip10pt
                           \eightrm#1 #2.\par}}
\def\U{\note{This code is used in}}
\def\ch{\note{\Y The following sections were changed by the change file:}
        \let\*=\relax}
\def\A{\note{See also}}




%---------------CODE LISTING--------------------------------------------

\def\P{\rightskip=0pt plus 100pt minus 10pt
  \sfcode`;=3000 \pretolerance 10000 \hyphenpenalty 1000 \exhyphenpenalty 10000
  \global\ind=2 \1\ \unskip}

\def\Q{\rightskip=0pt % leave code mode
  \sfcode`;=1500 \pretolerance 200 \hyphenpenalty 50 \exhyphenpenalty 50 }

\newcount\ind % current indentation in ems

\newbox\bak \setbox\bak=\hbox to -1em{}
\newbox\bakk\setbox\bakk=\hbox to -2em{}

\def\0{\ifmmode\ifinner$\par
  \hangindent\ind em\noindent\kern\ind em\ignorespaces$\fi
  \else\par
  \hangindent\ind em\noindent\kern\ind em\ignorespaces\fi}
\def\1{\global\advance\ind by1\hangindent\ind em}
\def\2{\global\advance\ind by-1}
\def\3#1{\hfil\penalty#10\hfilneg}
\def\4{\copy\bak}
\def\5{\hfil\penalty-1\hfilneg\kern2.5em\copy\bakk\ignorespaces}
\def\6{\ifmmode
       \else\par\hangindent\ind em\noindent
       \kern\ind em\copy\bakk\ignorespaces
       \fi}
\def\7{\Y\6}
\def\8{\unskip} % no indentation--works only in code, not in |...|
\def\9#1{{#1}}

% For comments

\def\C#1{\ifmmode\gdef\XX{\null$\null}\else\gdef\XX{}\fi
         \XX\hfil\penalty-1\hfilneg\quad
	 $\commentbegin\,${#1}$\,\commentend$\XX}
\def\B{\mathopen{\.{@\commentbegin}}}
\def\T{\mathclose{\.{@\commentend}}}

% For text in |...|

\def\CD{\relax\ifmmode\let\DC\egroup\hbox\bgroup\else\let\DC\relax\fi}
\let\DC=\relax

%---------------LaTeX SUBSTITUTIONS-------------------------------------

\let\tensy\relax
\let\sevenrm\relax
\def\writeconitem#1{}
\let\N\M

